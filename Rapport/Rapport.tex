\documentclass{article}

\usepackage[english]{babel}

\usepackage[letterpaper,top=2cm,bottom=2cm,left=3cm,right=3cm,marginparwidth=1.75cm]{geometry}
\usepackage{amsmath}
\usepackage{graphicx}
\usepackage{csquotes}
\usepackage[colorlinks=true, allcolors=blue]{hyperref}
\usepackage{biblatex} 

\addbibresource{sample.bib} %Import the bibliography file

\title{Rapport de projet}
\author{Eliott Rakotondramanitra, Raphaël Renard, Luc Salvon}

\begin{document}
\maketitle



\section{Introduction}

- Présentation du sujet (Algorithmes de renforcement génériques pour les jeux)

- Explications des deux jeux choisis ainsi que leurs types (plateau, course, états continus, états discrets…), on peut également expliquer que d'autres jeux ont été choisis mais non retenus et pourquoi.

- Lister les différents algorithmes de machine learning que l'on va implémenter.

- Présenter les différentes contraintes imposées par CodinGame.

- Explication du protocole expérimental (entraînement sur un environnement codé en interne puis déploiement sur CodinGame).


\section{Mad pod racing}
\subsection{Explications du jeu}
environnement, actions possibles, conditions de victoire, déroulement d'une partie...

\subsection{Présentation des algorithmes}
Pour chaque algo:

-	Explication simplifiée de l'algo

-	Comment sont représentés les états et les actions dans le jeu ?

-	Analyses des résultats en interne (graphes), des problèmes rencontrés, les solutions proposées qui ont marché ou non…

-	Comment a-t-on simplifié les modèles de RL et la structure du code pour envoyer sur CodinGame ? (Si c'est commun aux deux jeux on peut faire une partie commune « IV. Simplifier les modèles de RI » )

-	Comparaison entre les résultats en interne et les résultats sur CodinGame


\section{Ultimate Tic Tac Toe}
Pareil


\section{Conclusion}


\nocite{*}
\printbibliography 

\end{document}